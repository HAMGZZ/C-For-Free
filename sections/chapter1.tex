\section{Chapter 1: The Compiler \& Hello World}
C was created in 1970 by Dennis Ritchie during his time at Bell Labs as the implementation language of Unix. With it's success being put down to the popularity of Unix and in part as it's adoption of a standard programming language \cite{cHist}. To create a C program, one needs a compiler to change the code into machine code. There are many choices available such as MinGW, CLang and the original pcc (Portable C Compiler). However, this text will work purely with the GCC compiler as it has it's roots in the free C compiler of 1984 and is what Linux, *BSD and Apple's Xcode (until recently) uses to compile C - it is fair to say that GCC is one of the most used compilers today and throughout history.\\
\subsection{GCC on a system}
GCC is essentially a set of command line tool that can be installed on any common computer system. One will need to install GCC (binaries can be found here: \url{https://gcc.gnu.org/}) to follow on with this text.\\
Once GCC is installed on your host we can go through the basics.\\
\subsubsection{The GCC Command}
If we run just \texttt{gcc} we should see the following result:
\begin{commandline}
\begin{verbatim}
	$ gcc
  gcc: fatal error: no input files
  compilation terminated.
\end{verbatim}
\end{commandline}
Firstly, great job on getting GCC installed, secondly we now need to understand what it's asking for.\\
The great thing about GCC is that it gives a lot of hint's to what is wrong with what one may have entered. Luckily for us, it tells us we are missing an input file. From the GCC manual (or help in the command line) gcc requires the following: \texttt{gcc [options] file...}\\
One can run GCC as the following:
\begin{commandline}
\begin{verbatim}
	$ gcc HelloWorld.c
$
\end{verbatim}
\end{commandline}
This will compile the \texttt{HelloWorld.c} file and create a program file called \texttt{a.out} which can be run by your system. However, the program name \texttt{a.out} is not very nice. To change this, we can add an option to the GCC compiler. The option \texttt{-o} enables us to denote an output name for the compiled program.\\
For example:
\begin{commandline}
\begin{verbatim}
	$ gcc -o helloworld HelloWorld.c
$
\end{verbatim}
\end{commandline}
This will produce a program executable with the name \texttt{helloworld} which is a lot more user friendly than the previous output. 
\subsubsection{Hello World}
This will be a quick look into compiling your own program, not a detailed analysis of the code... that will come later.\\
You can use whichever text editor you want to create these programs. I recommend using VS Code or notepad++ for those who are new. However, as with anything to do with programming, it is all your choice, so choose something you are comfortable with. \\
Create a new file with the \texttt{HelloWorld.c} file name.\\
Enter in the following code:
\begin{file}[HelloWorld.c]
\begin{lstlisting}[style=CStyle]
#include <stdio.h>

int main()
{
  printf("Hello, World\r\n");
  return 0;
}
\end{lstlisting}
\end{file}
It does not matter if you do not understand the code at this point, we are just trying to compile something so save this file. Now head to the command line, compile and run the file.\\
You should expect something similar:
\begin{commandline}
\begin{verbatim}
	$ gcc -o helloworld HelloWorld.c
$ ./helloworld
  Hello, World
$  
\end{verbatim}
\end{commandline}
Well done, you have just compiled your own C program!
